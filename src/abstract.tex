
%%%%%%%%%%%%% Abstract

\chapter*{Abstract} Search engines have become an integral tool for information retrieval since the growth of the internet. However, despite its huge importance, many organisations still face the challenge of developing effective search engines for their users. This is a problem the University of Sydney is yet to resolve. Experience with the University’s current search engine shows that queries searched for often return irrelevant or unsatisfactory information, resulting in a poor user experience. Thus, there is reason to improve the University’s current search implementation.

To solve this challenge faced by many organisations, a number of search tools have been developed that focus on understanding the meaning of the query instead of searching for literal matches of words and variants. These applications are known as semantic search engines. In the last decade, the application of ontologies and knowledge graphs as an instrument for semantic searching has become increasingly prominent within industry. This approach typically involves building a knowledge graph from various information sources, and then developing a search engine that can query over the graph to find relevant information. Knowledge graphs however introduce their own set of challenges as automating a process to create an extensive, truthful graph from various information sources proves to be a difficult task.
