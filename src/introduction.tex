\chapter{Introduction}\label{cha:introduction}
%5 pages
\section{Motivation} 
Search engines have played an integral role in information retrieval since the development of digital information. Many organizations provide search engines on their public Web sites to allow both external and internal users to retrieve public information within their domain. Site search engines have a much smaller subset of information than Web search engines, allowing them to be more customized to the user and updated more frequently. Information retrieval (IR) systems. Have you ever experienced better results from searching on a Google than searching on the site's search engine? This was the experience with the University of Sydney's search engine. 


% summary of the problem
% why is the problem significant

\section{Scope}
Several factors, including relevance, coverage, freshness, response time, and user interface, should be considered when evaluating the effectiveness of a search engine. Among these, relevance is the most critical but often hardest to get right. (Reference: Tefko 1, 2). Accordingly, this objective of this research paper is to design and implement a search engine that delivers relevant documents for a given query on the University's site.

The University currently provides several search engines:
\begin{itemize}
    \item University
    \item Library
    \item Current Students
\end{itemize}

To design a search engine, it is important to note the key features that make that comprise the problem space. 
\begin{itemize}
    \item users: The University faces the challenge of accommodating both internal stakeholders (e.g. researchers, staff, students) and external stakeholders (e.g. future students, media). The search engine must be able to accommodate queries from students, researchers and staff, both current and future.
    \item data types: Documents types can range from contact info, images, videos, journals and unit outlines
    \item security: It is imperative that only users are only shown documents that they are permitted to have access to.
    \item 
\end{itemize}

Once a the search engine has been developed, its effectiveness needs to be measured to evaluate if the new model has any improved results. Evaluation method will be discussed in future chapters.
% how to measure the effectiveness of a search engine (method)?



\section{Outline} 
% defines the objective and sections of this paper


% Site search.

% The search experience on the University’s web site leaves much to be desired. In an attempt to solve this problem, this project will develop and test an ontology-driven semantic search engine for the University.

% The following tasks will be completed:
% • Review relevant literature, including previous research on academic search engines
% • Develop and validate an ontology
% • Develop and validate a knowledge graph, utilizing the University of Sydney’s data repository along with the ontology developed prior
% • Develop and test a semantic search engine, utilizing the knowledge graph developed prior
% • Evaluate the quality and performance of the search engine through experimentation
% • Present the results, findings, lessons learned and conclusion in a thesis report