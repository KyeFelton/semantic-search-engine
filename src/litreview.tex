\chapter{Literature review\label{cha:litreivew}}
%30 pages

\section{History of Site Search}
%should I talk about semantic matching? query reformation?
The first occurrences of site search had relied on literal matching of words determine relevance. This approach is referred to as lexical search. Documents would be broken down into words or variants (referred to as terms or bag-of-words), and then indexed based on the frequency of terms. Documents were matched and ranked to a query based on the degree of matching between terms. Although simple, this approach is fairly effective and is still prevalent in industry (\cite{Li2013a}) [131, 52, 6].



\section{Semantic Search}
% an overview of semantic searching against typical search
Instead of following a formula to match terms from queries to documents, semantic search aims to understand the meaning behind the query entered to deliver more relevant results. Meaning can involve understanding the user's intent and the contextual meaning of words. Consider the following example\dots

There are several approaches to developing a semantic search engine - semantic matching, knowledge graphs.

\section{Knowledge Graphs}
% an overview of how ontologies and knowledge graphs can help
The development of knowledge graphs to aid semantic searching has increased in popularity. The goal of knowledge graphs is to reduce noisy Internet content to machine-readable facts about entities and their associations. Instead providing a link to a relevant document, once a search engine understands what the user is looking for, it can utilize the knowledge graph to return information that centers around the entity queried. For example\dots

\subsection{Construction}
Knowledge graphs consist of entities, attributes, classes, constraints (optional) and associations. Entities 
\subsection{Curation}
\subsection{Searching}

\section{Evaluation}
\subsection{Limitations}